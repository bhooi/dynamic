
\paragraph{Time Series Anomaly Detection.} Numerous algorithms exist for anomaly detection in univariate time series~\cite{keogh2007finding}. For multivariate time series, LOF~\cite{breunig2000lof} uses a local density approach. Isolation Forests~\cite{liu2008isolation} partition the data using a set of trees for anomaly detection. Other approaches use neural networks~\cite{yi2017grouped}, distance-based~\cite{ramaswamy2000efficient}, and exemplars~\cite{jones2014anomaly}. However, none of these consider sensor selection. 

\paragraph{Anomaly Detection in Temporal Graphs.} \cite{akoglu2010oddball} finds anomalous changes in graphs using an egonet (i.e. neighborhood) based approach, while \cite{chen2012community} uses a community-based approach. \cite{mongiovi2013netspot} finds connected regions with high anomalousness. \cite{araujo2014com2} detect large and/or transient communities using Minimum Description Length. \cite{akoglu2010event} finds change points in dynamic graphs, while other partition-based~\cite{aggarwal2011outlier} and sketch-based~\cite{ranshous2016scalable} exist for anomaly detection. However, these methods require fully observed edge weights (i.e. all sensors present), and also do not consider sensor selection. 

\paragraph{Power Grid Monitoring.} A number of works consider the Optimal PMU Placement (OPP) problem~\cite{brueni2005pmu}, of optimally placing sensors in power grids, typically to make as many nodes as possible fully observable, or in some cases, minimizing mean-squared error. Greedy~\cite{li2011phasor}, convex relaxation~\cite{kekatos2012optimal}, integer program~\cite{dua2008optimal}, simulated annealing~\cite{baldwin1993power}, and swarm-based~\cite{chakrabarti2008pmu} approaches have been proposed. However, these methods do not perform anomaly detection. \cite{rakpenthai2007optimal,zhao2012pmu,magnago1999unified} consider OPP in the presence of branch outages, but not anomalies in general, and due to their use of integer programming, only use small graphs of size at most 60. 

\paragraph{Epidemic and Outbreak Detection.} \cite{leskovec2007cost} proposed CELF, for outbreak detection in networks, such as water distribution networks and blog data, also using a submodular objective function. Their setting is a series of cascades spreading over the graph, while our input data is time-series data from sensors at various edges of the graph. For epidemics, \cite{pastor2002immunization,cohen2003efficient} consider targeted immunization, such as identifying high-degree~\cite{pastor2002immunization} or well-connected~\cite{cohen2003efficient} nodes. We show experimentally that our sensor selection algorithm outperforms both approaches.

{
\aboverulesep=0ex
\belowrulesep=0ex
\renewcommand{\arraystretch}{1.1}
\begin{table}[h!]
\small
\centering
\caption{Comparison of related approaches: only \methodF satisfies all the listed properties.}
\label{tab:salesman}
\begin{tabular}{@{}rcccc|c@{}}
\toprule
 & \rotatebox{90}{\textbf{Time Series}~\cite{keogh2007finding,breunig2000lof}, etc.} & \rotatebox{90}{\textbf{Graph-based}~\cite{akoglu2010oddball,mongiovi2013netspot,shah2015timecrunch}} & \rotatebox{90}{\textbf{OPP}~\cite{brueni2005pmu,li2011phasor,kekatos2012optimal}, etc.} & \rotatebox{90}{\textbf{Immunization}~\cite{pastor2002immunization,cohen2003efficient}\ \ }  & {\bf \rotatebox{90}{\methodF}} \\ \midrule
\textbf{Anomaly Detection} & \Checkmark & \Checkmark & & & \CheckmarkBold \\ 
\textbf{Online Algorithm} & \Checkmark &  & & & \CheckmarkBold \\
\textbf{Using Graph Data} &  & \Checkmark & \Checkmark & \Checkmark & \CheckmarkBold \\ \midrule
\textbf{Sensor Selection} & &  & \Checkmark & \Checkmark & \CheckmarkBold \\ 
\textbf{With Approx. Guarantee} & &  & & & \CheckmarkBold \\ 
\bottomrule
\end{tabular}
\end{table}
}

Table \ref{tab:salesman} summarizes existing work related to our problem. \methodF differs from existing methods in that it performs anomaly detection using an online algorithm, and it selects sensor locations with a provable approximation guarantee. 

\subsection{Background: Submodular Functions}
A function $f$ defined on subsets of $\V$ is submodular if whenever $\T \subseteq S$ and $i \notin S$:
\begin{align}
f(\S \cup \{i\}) - f(\S) \le f(\T \cup \{i\}) - f(\T) 
\end{align}
Intuitively, this can be interpreted as \emph{diminishing returns}: the left side is the gain in $f$ from adding $i$ to $\S$, and the right side is the gain from adding $i$ to $\T$. Since $\T \subseteq S$, this says that as $\T$ `grows' to $\S$, the gains from adding $i$ can only diminish.

\cite{nemhauser1978analysis} showed that nondecreasing submodular functions can be optimized by a greedy algorithm with a constant-factor approximation guarantee of $(1-1/e)$. These were extended by \cite{sviridenko2004note} to the non-constant sensor cost setting. 
