Given sensor readings over time from a power grid consisting of nodes (e.g. generators) and edges (e.g. power lines), how can we most accurately detect when an electrical component has failed? More challengingly, given a limited budget of sensors to place, how can we determine where to place them to have the highest chance of detecting such a failure? Maintaining the reliability of the electrical grid is a major challenge. An important part of achieving this is to place sensors in the grid, and use them to detect anomalies, in order to quickly respond to a problem. Our contributions are: {\bf 1) Online anomaly detection:} we propose a novel, online anomaly detection algorithm that outperforms existing approaches. {\bf 2) Sensor placement:} we construct an optimization objective for sensor placement, with the goal of maximizing the probability of detecting an anomaly. We show that this objective has the property of submodularity, which we exploit in our sensor placement algorithm. {\bf 3) Effectiveness:} Our sensor placement algorithm is provably near-optimal, and both our algorithms outperform existing approaches in accuracy by $59\%$ or more (F-measure) in experiments. {\bf 4) Scalability:} our algorithms scale {\bf linearly}, and our detection algorithm is {\bf online}, requiring bounded space and constant time per update.